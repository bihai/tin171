\documentclass[a4paper,11pt]{article}
\usepackage{fancyhdr}
\usepackage{amsmath}
\setlength{\headheight}{11pt}
\pagestyle{fancyplain}
%\renewcommand{\chaptermark}[1]{\markboth{#1}{}}

\lhead{ }
\renewcommand{\headrulewidth}{0.0pt}

\lfoot{Group 1: Machine Learning}
%COMMENT: Y is the topic which has been assigned to you
\cfoot{\thepage}
\rfoot{}

%
%%    homebrew commands -- to save typing
\newcommand\etc{\textsl{etc}}
\newcommand\eg{\textsl{eg.}\ }
\newcommand\etal{\textsl{et al.}}
\newcommand\Quote[1]{\lq\textsl{#1}\rq}
\newcommand\fr[2]{{\textstyle\frac{#1}{#2}}}
\newcommand\miktex{\textsl{MikTeX}}
\newcommand\comp{\textsl{The Companion}}
\newcommand\nss{\textsl{Not so Short}}

\begin{document}
%-----------------------------------------------------------
\title{Recommender systems for movies using Collaborative Filtering}
\author{Group 1: Emil Falk, G\"oran Weinholt, Salvo Tomaselli, Rodolphe Lepigre}

\maketitle
%-----------------------------------------------------------
\begin{abstract}\centering
%%COMMENT
A recommender system for movies is a system that, based on users'
ratings of different movies, predicts what new and unrated
movies a user might like. The most common approach to this problem is
collaborative filtering. In contrast with content-based filtering, which
uses features extracted from the movies to make predictions, this method
looks at other users' preferences and ratings and tries to find users with
similar tastes. Most commonly a limited number of the best guesses
are then presented to the user. In this survey we look at relevant papers
that relate to this problem.
\end{abstract}

\section{Introduction}
\subsection{The problem or area you are interested in}
% Extract an accurate description of the problem from the project description or
% from your pre-proposal, or lift that material to a general problem specificat-
% ion.
A recommender system in general contains a user database,
an item database and a rating system. Each user might have different attributes
(age, gender, occupation, etc) and the same is true for the items (title, genre, director, etc).
The users can rate an item according to a predefined rating system and all
ratings are collected into the system.

The task of the recommender system is to find the unrated items that a user
might rate the highest in the future. This is done by using all the collective
information from the system (user data, item data and ratings).

The problem at hand is to build a recommender system that maximizes the usefulness
or utility towards its users.
\subsection{Significance}
% Why is the problem interesting or significant? What will a solution achieve?
The significance of this is gigantic in today's Internet based society. If you are running some sort
of service -- be it for movies, music or food etc -- then the sole purpose of that site probably is that
you want your users to somehow expand their knowledge in that area. Recommender systems gives us the
opportunity to supply the user with qualified guesses of which items the user might want to be
interested in.

If you are selling something the significance becomes even more apparent. If
you can provide customers with good targeted advertising for other items they might like then
you may very well be able to increase your profits.

\subsection{Aspects of the problem covered by this survey}
% Describe these briefly. You could write a small paragraph for each of them,
% highlighting keywords, and citing the relevant papers. The keywords could be
% used as sub-section headings in the survey proper, in section 3.
The area of recommender systems is roughly split into two different main approaches \cite{1423975}:
content-based filtering and collaborative filtering.

Content-based approaches will try to compute a list of recommendations based on previous user ratings.
It will try to find movies that are similar to other movies the user has liked in the past using some
similarity measure. This approach is dependent upon the content of the items. Collaborative filtering
on the other hand tries to find a list of recommendations for a user based on other users that have similar
rating behaviour. It is assumed that if their previous ratings are similar then unrated movies of one user
may very well be rated the same as by another user. This approach requires no information regarding the content
of the items but uses only relational data between the users.

Both approaches of course have their pros and cons but some that are rather important in our case are:
introducing \emph{serendipity} into the system; dealing with \emph{sparsity} of the ratings data; the
\emph{new user problem}; and the \emph{cold start problem} (which is specific to collaborative filtering).

Collaborative filtering is in turn split in two main parts: \emph{neighbourhood} or \emph{model-based} methods.

\subsection{Important aspects of the problem not covered here}
% Aspects of the problem one must know to understand it completely, but not
% included in the discussion. Use this subsection to ensure a good coverage of
% the relevant literature without going through much detail.
There is a third approach called Hybrid systems which combine the two previously stated approaches into a new,
more effective and powerful solution. This will not be covered in this survey.

\section{Framework}
% Here you describe which solution approach(es) you would have chosen if this
% would have been your course project. Use your own words, write down
% definitions, etc., )
\subsection{Define formally an instance of the problem}
% Either mathematically, or using pseudo-code. Also, give an example.
$U : \{u_1, u_2, ..., u_n\}$ be the set of all users\\
$M : \{m_1, m_2, ..., m_l\}$ be the set of all movies\\
$S : \{1,2,3,4,5\}$ be the set of all possible ratings \\
$R : \{r_{um} : \forall u \in U, \forall m \in M\}$ be the set of all ratings by all users\\
$M_i : \{m_1, m_2, ..., m_k\}$ be the set of movies that user $i$ has rated \\

Thus rating $r_{ij}$ is the rating of user $i$ on movie $j$. Each rating can either be undefined
or one of the elements of $S$. Then let: \\

$f(u,m) : U \times M \to S$ \\

be a function that predicts user $u$'s rating on movie $m$.

With this function $f$ we want to find a movie $i$ such that $i \in M \setminus M_i$.
We use function $f$ to find the best movie recommendation $i^{*}$ to
the user $u$ such that: \\
\[ i^{*} = \operatorname*{arg\,max}_{j \in M_i \setminus M} f(u,j) \]

This is the formal definition of a recommendation problem. But there is a central problem. The
function $f$ id not defined over the whole set $U \times M$. Thus we will need to estimate
the missing ratings. This estimation is done using collaborative filtering techniques.

We seldom want to recommend only one item and thus this problem is commonly generalized to
to presenting a list of the $N$ best movie recommendations.

\subsection{Benchmarks}
% How would you measure the performance of a program that solves the problem?
% Find instances/benchmarks, or make your own).
The performance of a recommender system can be evaluated using ordinary machine learning techniques.
This involves splitting our set of ratings into two sets: $R_{train}$ and $R_{test}$. To test our system
we train it using the set $R_{train}$. When we have a trained system we measure its error on our test set
$R_{test}$. Two ways of measuring the magnitude of the error are \emph{Mean Absolute Error} (MAE) and
\emph{Root Mean Squared Error} (RMSE) \cite{springerlink:10.1007/978-0-387-85820-3_1}.
Furthermore they show how to divide our rating set into two ``good'' sets in \cite{Herlocker:2004:ECF:963770.963772},
including other various things to keep in mind when evaluating the performance of our system.

Both papers also mention two other central metrics: the \emph{precision}
measures the probability that an item is considered as relevant and the
\emph{recall} measures the probability that an item is recommended \cite{Herlocker:2004:ECF:963770.963772}.

\section{Survey}
\subsection{Results from the literature}
% Describe briefly the scientific papers you found relevant to the problem.
% Ideally, a commented summary of each paper, like an extended annotation.
% Cite references. You could organize this section by keyword, one sub-section
% for each keyword you specified in 1.3.
\subsubsection{Collaborative filtering}
Collaborative filtering is one of the most used methods to predict the ratings in
a recommender system. It is covered extensively in both \cite{springerlink:10.1007/978-0-387-85820-3_1} and \cite{1423975}. This method has the advantage over content-based approaches that it need not care about the content of the recommended items \cite{1423975}. It only needs rating data from the users to work properly. This approach is in turn divided into two sub-approaches: neighbourhood- and model-based methods. Neighbourhood methods have been used a lot in the past but today more complex model-based methods are often used such as \emph{Matrix Factorization}.
%% TODO: ADD SOMETHING ABOUT ITEM-ITEM/USER-USER systems

\subsubsection{Serendipity}
One of the biggest problems with content-based filters is that they are very na\"ive. If a user $A$ has given high ratings to a lot of Quentin Tarantino movies then the recommender system will probably recommend to $A$ the rest of his movies. But since $A$ already liked Tarantino it is rather likely that the user would have found the recommended movies by themselves. This is the problem of low serendipity in content-based systems.

Collaborative filtering on the other hand will have good serendipity because it will not be influenced in this way. Of course there may be a lot of users that are Quentin Tarantino fans, but they will also have other movies rated, which will in this scenario be good candidates to recommend to the user $A$.

\subsubsection{Sparsity}
Another problem when dealing with recommender system is that our data is often
very sparse. A user might only have rated a few percent, if even that, of the total
number of items. But since this is a common scenario we have to be able to make
good predictions even in the presence of sparse data. Data sparsity is a big problem when using \emph{Neighbourhood methods} \cite{springerlink:10.1007/978-0-387-85820-3_1}. In both \cite{1423975} and \cite{springerlink:10.1007/978-0-387-85820-3_1} they suggest a few techniques to overcome this problem. A simple and common technique is to replace undefined values with a default value. The most promising seems to be a slightly modified version of \emph{Singular Value Decomposition} (SVD+), a common method for \emph{Matrix Factorization}.

\subsubsection{New user problem}
The new user problem exists in both content-based filtering and collaborative filtering. For collaborative filtering the problem is that when new users enter the system they don't have any ratings and it is therefore impossible to find users with similar rating behaviour and thus no recommendations can be made \cite{1423975}. Both \cite{1423975} and \cite{springerlink:10.1007/978-0-387-85820-3_1} explain methods for overcoming this problem. Most of the solutions are to make a hybrid system. If we use collaborative filtering and a user has no ratings then we could revert to content-based filtering.

\subsubsection{Cold start problem}
The cold start arises in most collaborative filtering systems. The problem is that when the system is rather new there are a lot of items that are unrated. In a system using collaborative filtering unrated items will never be recommended.

\subsubsection{Neighbourhood methods}
Neighbourhood methods are one of two main approaches to collaborative filtering. It's a very simple and elegant solution to this problem. The method tries to cluster the users into neighbourhoods using some distance measure. The recommender system then recommends items unrated by the user in its neighbourhood.
%% TODO: the ``this problem'' reference is unclear (what problem? cold start?)
%% TODO: the previous sentence makes no sense

These methods have previously been very widely used due to their simplicity and rather good performance. But in recent years they have seen a decline in favor of model-based methods.

\subsubsection{Model-based methods}


\subsection{Tools and programs available for the problem, or for closely related ones}
% Describe these briefly. Say how they can be used, and how a solution to your
% problem can build on them, or differ from them.
There exist many commercial and free tools for doing recommender systems in various domains, mostly targeting e-commerce.

MATLAB \cite{matlab-ml} is by many considered
the number one tool for doing all kinds for machine learning related engineering. It has built-in support for a lot of the constructs used in
machine learning and statistics. A free alternative to MATLAB is GNU Octave \cite{octave}. It doesn't have as complete a feature set
as MATLAB, but it's more than enough for most cases.

Shogun \cite{shogun} is a toolbox for machine learning tasks that contains implementations of many relevant algorithms. It interfaces with both MATLAB
and GNU Octave.

%%\chapter{Appendix}
%%The previous sections (excluding title page and references) are limited to 6 pages. Here, you can put in anything that didn't fit in there, but that you still think is essential.

\bibliographystyle{aiaa}
\bibliography{refs}

\end{document}
