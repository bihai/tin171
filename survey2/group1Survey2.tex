\documentclass[a4paper,11pt]{report}
\usepackage[utf8]{inputenc}
\usepackage{fancyhdr}
\usepackage{algpseudocode}
\setlength{\headheight}{11pt}
\pagestyle{fancyplain}
\renewcommand{\chaptermark}[1]{\markboth{#1}{}}

\lhead{ }
\renewcommand{\headrulewidth}{0.0pt}

\lfoot{Group 1: Machine Learning}
%COMMENT: Y is the topic which has been assigned to you
\cfoot{\thepage}
\rfoot{}

%
%%    homebrew commands -- to save typing
\newcommand\etc{\textsl{etc}}
\newcommand\eg{\textsl{eg.}\ }
\newcommand\etal{\textsl{et al.}}
\newcommand\Quote[1]{\lq\textsl{#1}\rq}
\newcommand\fr[2]{{\textstyle\frac{#1}{#2}}}
\newcommand\miktex{\textsl{MikTeX}}
\newcommand\comp{\textsl{The Companion}}
\newcommand\nss{\textsl{Not so Short}}

\begin{document}
%-----------------------------------------------------------
\title{Survey: Recommender systems for movies}
\author{Group : Emil Falk, Göran Weinholt, Salvo Tomaselli, Rodolphe Lepigre}

\maketitle
%-----------------------------------------------------------
\begin{abstract}\centering
%%COMMENT
no more than 10 lines!
\end{abstract}

\chapter{Introduction}
\section{Recommender Systems}
A recommender system is a system that based on some data makes predictions on what a user might
like and present these predictions to the user. Lets say you run a community for movies like IMDb.
Every user has the possibility to rate different movies to their preference on a given scale. Based
on the users ratings and information about all the movies the recommender system tries to find
qualified guesses about which new movies to present to the user as good recommendations.

\section{Significance}
The significance of this is gigantic in todays internet-based society. If you are running some sort 
of service, be it for movies, music or food etc. Then the sole purpose of that site probably are that
you want your users to expand their knowledge in that area somehow. Recommender Systems gives us the
opportunity to supply the user with qualified guesses of what items the user might want to observe next.

If you on the other hand are selling something the significance becomes even more apparent. If
you can provide a buyer with other items that it might like then you may very well be able to increase
your profits.

A recommender system is today almost standard on all large e-commerce sites and large content-based 
communities like IMDb, Last.fm etc.

\section{Aspects of the problem covered by this survey}
The area of Recommender systems are roughly split into two different main approaches: Content-based filtering and
Collaborative filtering. This is also what will be covered in this survey. As stated in NEEDREF the most successful
approach to movie recommendation has been using collaborative filtering techniques. Thus most of the survery will
cover collaborative filtering. A brief text on content-based filtering in this setting can be found in \ref{sec:results}.

We will describe how to model our data, extract features and learn the parameters to our prediction model.
Keywords:
Content-based filtering
Collaborative filtering
Data model
Feature space
Fitting
Learning methods:
  Neighbourhood methods
  Matrix factorization

\section{Important aspects of the problem not covered here}
There is a third approach called Hybrid systems which combine the two previously stated approaches into a new,
more effective and powerful solution. This will not be covered in this survey.

\chapter{Framework}

\section{Define formally an instance of the problem}
We have a database of users and their ratings from a scale 1-5.
We have a database of movies.

SOME MATH.

\section{Benchmarks}
Measure performance, two refs about performance
How does one measure performance? Cross-validation, prediction accuracy etc.

\chapter{Survey}
\section{Results from the literature}
\label{sec:results}
\subsection{Data model}
Something about how to model the data

\subsection{Feature space}
Something about how to extract good features from our data. e.g genre, age-rating, director etc.

\subsection{Content-based filtering}
Something small about content-based filtering. Pros, cons.

\subsection{Collaborative filtering}
Something a bit bigger about collaborative filtering.
\subsubsection{Neighbourhood methods}
Something about early neighbourhood methods
\subsubsection{Matrix factorization}
Something about the more recent approach of matrix factorization

\section{Tools and programs available for the problem, or for closely related ones}
There exist several commercial tools for doing recommender systems, mostly targeting e-commerce. MATLAB\cite{matlab-ml} are by many considered
the number one tool for doing all kinds for machine learning-related engineering. It has built-in support for alot of the constructs used in 
machine learning and statistics. A free alternative to MATLAB are Octave\cite{octave} that is released under GNU. It does not have as good support
as MATLAB but more than enough for most cases.

Shogun\cite{shogun} is a toolbox for machine learning tasks contaning implementation of alot of relevant algorithms. It interfaces to both MATLAB
and Octave to name a few.

\chapter{Appendix}
The previous sections (excluding title page and references) are limited to 6 pages. Here, you can put in anything that didn't fit in there, but that you still think is essential.

\bibliographystyle{aiaa}
\bibliography{refs}

\end{document}
