\documentclass[a4paper,11pt]{article}
\usepackage{fancyhdr}
\setlength{\headheight}{11pt}
\pagestyle{fancyplain}
%\renewcommand{\chaptermark}[1]{\markboth{#1}{}}

\lhead{ }
\rhead{}
%\renewcommand{\headrulewidth}{0.0pt}

\lfoot{Group 1: NLP1 survey}
%\lfoot{Group X: Y survey}
%COMMENT: Y is the topic which has been assigned to you
\cfoot{\thepage}
\rfoot{}

%
%%    homebrew commands -- to save typing
\newcommand\etc{\textsl{etc}}
\newcommand\eg{\textsl{eg.}\ }
\newcommand\etal{\textsl{et al.}}
\newcommand\Quote[1]{\lq\textsl{#1}\rq}
\newcommand\fr[2]{{\textstyle\frac{#1}{#2}}}
\newcommand\miktex{\textsl{MikTeX}}
\newcommand\comp{\textsl{The Companion}}
\newcommand\nss{\textsl{Not so Short}}


\begin{document}
%-----------------------------------------------------------
\title{Natural language processing\\
in a dialog system for a student help-desk} % TODO title ?
\author{Group 1: Emil Falk, Rodolphe Lepigre,\\
        Salvatore Tomaselli, G\"oran Weinholt}
\maketitle
%-----------------------------------------------------------
\begin{abstract}\centering
%%COMMENT
Dialog systems are AI systems in which an agent is capable of interaction with
an user through natural language. We give some hints of how to build such a
system in the context of a student help-desk. More precisely, this system
should allow an user (supposedly a student) to ask specific questions through
a terminal, and get useful answers back from the system. To be able to build
such a system we will use tools like the Grammatical Framework (GF) that is
described in \cite{ranta-2011}.
\end{abstract}

\section{Introduction}
\subsection{The problem or area you are interested in}
% Extract an accurate description of the problem from the project description or
% from your pre-proposal, or lift that material to a general problem specificat-
% ion.
Natural language processing is a very wide topic. In this survey we only
consider written natural language processing, but there exists plenty softwares
that can recognise spoken language, or the mood of the user through its
intonation. In our case, the user will interact with the machine by typing
questions into a terminal using a ``chat'' format. The program will then parse
the question and analyse its content in order to give an answer that is as
relevant as possible.

\subsection{Significance}
% Why is the problem interesting or significant? What will a solution achieve?
This problem seems interesting to us because it would give an alternative to
the usual FAQ interface that most websites offer. The student help-desk example
seems appropriate since there are plenty of data that we can use as a knowledge
base for out system.

\subsection{Aspects of the problem covered by this survey}
% Describe these briefly. You could write a small paragraph for each of them,
% highlighting keywords, and citing the relevant papers. The keywords could be
% used as sub-section headings in the survey proper, in section 3.)
% TODO
% I think we should only consider the grammar part, and a very simple grammar
% (that does not allow a lot of questions).

\subsection{Important aspects of the problem not covered here}
% Aspects of the problem one must know to understand it completely, but not
% included in the discussion. Use this subsection to ensure a good coverage of
% the relevant literature without going through much detail.
% TODO
% There we should talk about "syntax understanding", how the program "underst-
% ands" a question.

\section{Framework}
% Here you describe which solution approach(es) you would have chosen if this
% would have been your course project. Use your own words, write down
% definitions, etc., )
% TODO
% Here description of GF + a few things about the Haskell interface (abstract
% syntax). How does it work together with the parser, ...

\subsection{Define formally an instance of the problem}
% Either mathematically, or using pseudo-code. Also, give an example.
% TODO
% This should be easy, just simulate a "chat" between the program and an user.

\subsection{Benchmarks}
% How would you measure the performance of a program that solves the problem?
% Find instances/benchmarks, or make your own).
% TODO
% I don't really know what should be here...

\section{Survey}
\subsection{Results from the literature}
% Describe briefly the scientific papers you found relevant to the problem.
% Ideally, a commented summary of each paper, like an extended annotation.
% Cite references. You could organize this section by keyword, one sub-section
% for each keyword you specified in 1.3.
% TODO
% I still have to find some references.

\subsection{Tools and programs available for the problem, or for closely related ones}
% Describe these briefly. Say how they can be used, and how a solution to your
% problem can build on them, or differ from them.
% TODO
% Obviously some stuff about GF, and it would be nice to talk about other
% sytems too.

% \section{References}
\bibliographystyle{aiaa}
\bibliography{refs}

% \section{Appendix}
% The previous sections (excluding title page and references) are limited to 6
% pages. Here, you can put in anything that didn't fit in there, but that you
% still think is essential.
% TODO
% We probably won't need an apendix.
\end{document}
