\documentclass[a4paper,11pt]{article}
\usepackage{fancyhdr}
\setlength{\headheight}{11pt}
\pagestyle{fancyplain}
%\renewcommand{\chaptermark}[1]{\markboth{#1}{}}

\lhead{ }
\rhead{}
%\renewcommand{\headrulewidth}{0.0pt}

\lfoot{Group 1: NLP1 survey}
%\lfoot{Group X: Y survey}
%COMMENT: Y is the topic which has been assigned to you
\cfoot{\thepage}
\rfoot{}

%
%%    homebrew commands -- to save typing
\newcommand\etc{\textsl{etc}}
\newcommand\eg{\textsl{eg.}\ }
\newcommand\etal{\textsl{et al.}}
\newcommand\Quote[1]{\lq\textsl{#1}\rq}
\newcommand\fr[2]{{\textstyle\frac{#1}{#2}}}
\newcommand\miktex{\textsl{MikTeX}}
\newcommand\comp{\textsl{The Companion}}
\newcommand\nss{\textsl{Not so Short}}


\begin{document}
%-----------------------------------------------------------
\title{Natural language processing\\
in a dialog system for a student help-desk}
\author{Group 1: Emil Falk, Rodolphe Lepigre,\\
        Salvatore Tomaselli, G\"oran Weinholt}
\maketitle
%-----------------------------------------------------------
\begin{abstract}\centering
%%COMMENT
Dialog systems are AI systems in which an agent is capable of interaction with
a user through natural language. We give some hints of how to build such a
system in the context of a student help-desk. More precisely, this system
should allow a user (i.e.~a student) to ask specific questions, and get
useful answers back from the system.
\end{abstract}

\section{Introduction}
\subsection{The problem or area you are interested in}
% Extract an accurate description of the problem from the project description or
% from your pre-proposal, or lift that material to a general problem specificat-
% ion.
Natural language processing is a very wide topic. In this survey we only
consider written natural language processing, but there exists plenty of
softwares that can recognise spoken language, or the mood of the user through
intonation or gestures. In our case, the user will interact with the machine
by typing questions into a terminal. The program will then parse the question
and analyse its content in order to give an answer that is as relevant as
possible. The domain of the question that the system should be able to
answer is restricted to the domain of a student help-desk.

\subsection{Significance}
% Why is the problem interesting or significant? What will a solution achieve?
This problem seems interesting to us because it would give an alternative to
the usual FAQ interface that most websites offer. The student help-desk example
seems appropriate since there are plenty of data that we can use as a knowledge
base for our system. Moreover, as students, we know how difficult it can be
to find answers to questions related to administrative matters.

\subsection{Aspects of the problem covered by this survey}
% Describe these briefly. You could write a small paragraph for each of them,
% highlighting keywords, and citing the relevant papers. The keywords could be
% used as sub-section headings in the survey proper, in section 3.
In this survey we will emphasize parsing natural language, and generating
natural language. In other words we will talk about the interface between the
machine and the human user, which is a very important part of natural language
processing. It seems important to recall that we will consider a closed-domain
system.

\subsection{Important aspects of the problem not covered here}
% Aspects of the problem one must know to understand it completely, but not
% included in the discussion. Use this subsection to ensure a good coverage of
% the relevant literature without going through much detail.
We will not discuss much the middle part of our question answering system, which
is the choice of the information to give to the user given an abstract syntax of
the language. However we will give some references to research articles that
cover those. We do not discuss open-domain systems that seem to be important in
the current research areas. This can be explained by the use of the huge amount
of data that is offered by the Internet. Such systems are discussed in
\cite{designinganopendomainqasystem}.

\section{Framework}
% Here you describe which solution approach(es) you would have chosen if this
% would have been your course project. Use your own words, write down
% definitions, etc., )
In order to parse and generate a natural language, we would use the Grammatical
Framework (GF) described in \cite{ranta-2011}. This seems like a good idea since
once we have designed a grammar, it is possible to make it support different
languages quite easily.

For the information choosing part, there are several possibilities, but since
we only process questions, we can get a lot of information from the syntax. For
example, if the question starts with the word "where", we know that the user is
looking for some place (a room, or a building). It significantly reduced the set of
possible answers. With a well-structured knowledge-base, and some machine
learning based on a user feedback system, it should be possible to get a robust
system.

\subsection{Define formally an instance of the problem}
% Either mathematically, or using pseudo-code. Also, give an example.
A student, lets call him Bob, is looking for information about the application
procedure to a Master program at Chalmers. He types the URL of the student help
desk system in his browser. A web page opens with a text-field where he can ask
his question. Let's say that his question is ``What is the application procedure
for a Master program?''. The system should give him an answer directing him to
the admission system. If the system cannot find an answer to the query, then it
must notify the user and ask him for a new question.

\subsection{Benchmarks}
% How would you measure the performance of a program that solves the problem?
% Find instances/benchmarks, or make your own).

There is no easy way to automatically benchmark a program solving this
problem. In fact, if it were possible to evaluate the relevance of an
answer in a systematic way, we would have solved the problem exactly.
A way to get an idea of how well the system works would be to include
a user feedback system, asking the user if the answer was useful and
relevant. 

In \cite{rankingsuspected} they have data on which answers are
correct. They find out how the system ranked the correct answer
and merely take the reciprocal of that. This can then be averaged
over several answers to get a mean performance.

\section{Survey}
\subsection{Results from the literature}
% Describe briefly the scientific papers you found relevant to the problem.
% Ideally, a commented summary of each paper, like an extended annotation.
% Cite references. You could organize this section by keyword, one sub-section
% for each keyword you specified in 1.3.
The course book \cite{aimodern} contains only a little paragraph on question
answering systems. However, it gives some hint on how to design natural language
processing (NLP) systems in general.

There are several papers that consider feeding data to a question system through
frequently asked question (FAQ) data (see \cite{evaluatingqasusingfaq} and
\cite{faqfindersys}).

In some cases the system might have to choose between several possible answers.
Ranking techniques for suspected answers to questions are discussed in
\cite{rankingsuspected}. However this is not a crucial matter for closed-domain
question answering systems.
% TODO add some more, maybe find new ones

\subsection{Tools and programs available for the problem, or for closely related ones}
% Describe these briefly. Say how they can be used, and how a solution to your
% problem can build on them, or differ from them.
The tool that seems the most attractive to us is the Grammatical Framework (GF). It
is described in \cite{ranta-2011}. When a grammar is created for a choosen domain,
you get a parser and a natural language generator for free. Support for several
languages is also possible without too much trouble. Natural language is interfaced to the
Haskell language, for example, through an abstract syntax tree on which one can work
to find out the meaning of a question.

In this project, we do no plan to support spoken language support. It would be
possible to use Olympus (\cite{olympus}), which is a complete framework for
implementing spoken dialog systems, to this end.

% \section{References}
\bibliographystyle{aiaa}
\bibliography{refs}

% \section{Appendix}
% The previous sections (excluding title page and references) are limited to 6
% pages. Here, you can put in anything that didn't fit in there, but that you
% still think is essential.
\end{document}
