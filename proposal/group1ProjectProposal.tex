\documentclass[a4paper,11pt]{article}
\usepackage{fancyhdr}
\setlength{\headheight}{11pt}
\pagestyle{fancyplain}
\usepackage{algpseudocode}
%\renewcommand{\chaptermark}[1]{\markboth{#1}{}}

\lhead{ }
\rhead{}
%\renewcommand{\headrulewidth}{0.0pt}

\lfoot{Group 1: S1 proposal}
%COMMENT: Y is the topic which has been assigned to you
\cfoot{\thepage}
\rfoot{}

%
%%    homebrew commands -- to save typing
\newcommand\etc{\textsl{etc}}
\newcommand\eg{\textsl{eg.}\ }
\newcommand\etal{\textsl{et al.}}
\newcommand\Quote[1]{\lq\textsl{#1}\rq}
\newcommand\fr[2]{{\textstyle\frac{#1}{#2}}}
\newcommand\miktex{\textsl{MikTeX}}
\newcommand\comp{\textsl{The Companion}}
\newcommand\nss{\textsl{Not so Short}}


\begin{document}
%-----------------------------------------------------------
\title{Fun With Chinese Checkers}
\author{Group 1: Emil Falk, Rodolphe Lepigre, Salvatore Tomaselli, G\"oran Weinholt}
\maketitle
%-----------------------------------------------------------
\begin{abstract}\centering
%%COMMENT

Chinese Checkers is a board game for up to six players. The fact that
it's a multi-player game makes it quite different from more widely
explored two-player games like Chess. Writing a computer player for
Chinese Checkers will allow us to explore search algorithms and
heuristics. We hope to discover what makes a multi-player game
different from a two-player game and what effect this has on the
quality of a computer player.

\end{abstract}

\section{Introduction}
\subsection{What is the problem you are trying to solve?}

We want to write a computer program that plays a game. Chinese
Checkers is a board game that can be played by two to six players.
Each player controls ten pegs that are placed in the corners of a
hexagonal star. The board has little holes in it where the pegs can be
placed. In each round of the game a player must move a peg in one of
two ways. A peg can be moved to one of the six holes next to it or, if
one of the holes is occupied by another peg, it can also ``jump'' over
that peg in a straight line (thereby traveling two holes over). The
player can jump over other pegs an unlimited number of times every
round, potentially moving a peg straight across the whole board. The
objective is to move all pegs to the opposing corner.

\subsection{What results are available in the literature on that problem?}
%(describe briefly the scientific papers you found relevant to the problem, references to sec 4.)

Sturtevant has explored the max$^n$ and paranoid algorithms in
relation to Chinese Checkers
\cite{springerlink:10.1007/978-3-540-40031-8_8}. Later on he also
evaluated the UCT algorithm
\cite{springerlink:10.1007/978-3-540-87608-3_4}. The algorithms were
pitted against each other and it is not clear how well they play
against humans, but we will assume that they are adequate.

An interesting point that Sturtevant makes is that multi-player games
(i.e.~Chinese Checkers) are difficult for computers for two reasons:
search strategies are less effective than for two-player games and
there is a need for opponent modelling which is normally not required.

Research into programs that play games is likely to be geared towards
producing strong programs that one day will be able to beat the best
human players. Perhaps a weaker program would be just as suitable for
our needs. Huang describes a contest in which he supervised students
who wrote programs to play Chinese Checkers
\cite{Huang:2001:SGP:378593.378708}. The students used
iterative-deepening search and experimented with heuristics.

\subsection{What tools and programs are already available for the problem, or for closely related ones?}
%(describe these briefly. Say how you can use them, and how your work will build on them, or differ from them). 

\emph{cheech} is a free software implementation of Chinese Checkers
\cite{cheech}. It provides a graphical interface and a network
protocol. It appears that we can make our own program connect to
\emph{cheech} as a player and thereby use the existing multi-player
functionality and graphics.

\section{Key ideas and delimitations} 
\subsection{Why is the problem interesting or significant?  What will a solution achieve?}

An interesting aspect of Chinese Checkers is that it's a multi-player
game. This means that players can form temporary coalitions against
other players. While traditional two-player strategies can be used to
play the game they are not going to be able to play it perfectly.

\subsection{The central idea in your solution}

Our solution will consist of an implementation of \emph{iterative
  deepening depth-first-search} (IDDFS) \cite{aimodern}. This will be
combined with a heuristic to form a kind of informed search. To
improve the depth of our solution we will try and do some pruning of
``bad'' plys, e.g.~those that do not move the player any closer to the
goal state.

\subsubsection{Why do you think it will work?}

Our solution will work because we think that \emph{IDDFS} closely
relates to how a human might play the game: first evaluating
moves that are easily visible on the board and iteratively looking
deeper (further) into what might happen.

Even though \cite{Huang:2001:SGP:378593.378708} does not report on the
performance of \emph{IDDFS} we know that it at least has been used to
some degree of success. The performance of our solution will of course
depend heavily on finding good heuristics.

\subsection{Define an instance of the problem. How will you measure the performance of your program?} 
%(You must know this in advance, find instances /benchmarks, or make your own).

The program should play a game in the same way a human would play: the
program waits its turn and makes a move. The initial board setup is
one instance of the problem, but from the perspective of analysing the
game every intermediate state of the board may also be considered an
instance.

The \emph{cheech} program includes a number of bots that we can have
play against our program. The program that wins the most games will
clearly be the better program. Another metric, which Sturtevant used,
is to count the number of moves a player is away from the goal state.

\subsection{The scope of your work} 
%(also, what interesting and related things are outside the scope of your proposal?)

The scope of our project will be to try and challenge a set of amateur
human players. In developing the program we will explore how the
dynamics of the game changes in multi-player games. Playing a perfect
game must be outside our scope because it appears to be an open
research problem.

\section{First results}
%(A small piece of pseudo-code for a key part of the solution. )

\section{Iterative deepening depth-first-search}
A central part of our solution will be the \emph{IDDFS} algorithm.
Shown below is the 
high-level pseudocode for that algorithm \cite{aimodern}.\\
\begin{algorithmic}
\Function{ITERATIVE-DEEPENING-SEARCH}{\emph{problem}}
\For{\emph{depth} $= 0 \to \infty$}
\State \emph{results} $\gets$ DEPTH-LIMITED-SEARCH(\emph{problem, depth})
\If {\emph{results} $\not=$ \emph{cutoff}}
\State \Return \emph{results}
\EndIf
\EndFor
\EndFunction
\\
\Function{DEPTH-LIMITED-SEARCH}{\emph{problem, limit}}
\State \Return RECURSIVE-DLS(MAKE-NODE(\emph{problem}, INITIAL-STATE), \\
\hspace{50mm} \emph{problem, limit})
\EndFunction
\\
\Function{RECURSIVE-DLS}{node, problem, limit}
\If{\emph{problem}.GOAL-TEST(\emph{node}.STATE)}
\State \Return SOLUTION(\emph{node})
\ElsIf{\emph{limit} = 0}
\State \Return \emph{cutoff}
\Else
\State \emph{cutoff\_occured?} $\gets$ \emph{false}
\ForAll{\emph{action} in \emph{problem}.ACTIONS(\emph{node}.STATE)}
\State \emph{child} $\gets$ CHILD-NODE(\emph{problem, node, action})
\State \emph{result} $\gets$ RECURSIVE-DLS(\emph{child, problem, Emit - 1})
\If{\emph{result} = \emph{cutoff}}
\State \emph{cutoff\_occured?} $\gets$ \emph{true}
\ElsIf{\emph{result} $\not=$ \emph{failure}}
\State \Return \emph{result}
\EndIf
\EndFor
\If{\emph{cutoff\_occured?}}
\Return \emph{cutoff}
\Else
\ \Return \emph{failure}
\EndIf
\EndIf
\EndFunction
\end{algorithmic}

\bibliographystyle{aiaa}
\bibliography{refs}
% \section{References}

% Reminder: References should be cited (i.e. actually be referred to) at the
% appropriate place in your text, they should visibly in�uence your document,
% and they should convey as much information as possible to the reader.

% To help you produce a good literature survey, we advise you to use an annot-
% ated list of references.

% A reference to a paper will start with a unique ID (numbers, letters, ...)
% and have the following three paragraphs:
% \begin{enumerate}
% \item Author(s) [year of publication], ``Title of paper'', Journal (or conference etc.), location data (i.e., volume numbers, etc. needed to �nd the item), a live www link if available.
% \item A couple of lines saying what question the paper is attacking, and what the proposed solution is. Or what topic the reference deals with (your textbook has excellent examples in the sections entitled ``Bibilographical and Historical notes''). 
% \item A pointer back to where you cite this reference. 
% \end{enumerate}

% An example would look like this:
% \newline
% [REF1]
% \begin{enumerate}
% \item Turing, A.M. (1936).`On Computable Numbers, with an Application to the Entscheidungsproblem". Proceedings of the London Mathematical Society. 2 42: 230�65.
% \item The paper is about ... 
% \item The most important reference to this paper in our document is in Section XYZ
% \end{enumerate}

% A reference to a specific topic in a book (or large survey article) has similar 2nd. and 3rd. paragraphs, but the first will look like this:
% \begin{enumerate}
% \item Author(s) [year of publication], ``Title of book'', Publisher, a live www link if available.
% \textbf{Section numbers or page numbers in the book dealing with the material you are referring to}. 
% \end{enumerate}

% \section{Appendix}
% The first 3 sections (excluding title page and references) are limited to 8 pages. Here, you can put in anything that didn't fit in there, but that you still think is essential.
\end{document}
