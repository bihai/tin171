\documentclass[a4paper,11pt]{article}
\usepackage{fancyhdr}
\setlength{\headheight}{11pt}
\pagestyle{fancyplain}
\usepackage{algpseudocode}
%\renewcommand{\chaptermark}[1]{\markboth{#1}{}}

\lhead{ }
\rhead{}
%\renewcommand{\headrulewidth}{0.0pt}

\lfoot{Group 1: S1 proposal}
%COMMENT: Y is the topic which has been assigned to you
\cfoot{\thepage}
\rfoot{}

%
%%    homebrew commands -- to save typing
\newcommand\etc{\textsl{etc}}
\newcommand\eg{\textsl{eg.}\ }
\newcommand\etal{\textsl{et al.}}
\newcommand\Quote[1]{\lq\textsl{#1}\rq}
\newcommand\fr[2]{{\textstyle\frac{#1}{#2}}}
\newcommand\miktex{\textsl{MikTeX}}
\newcommand\comp{\textsl{The Companion}}
\newcommand\nss{\textsl{Not so Short}}


\begin{document}
%-----------------------------------------------------------
\title{Fun With Chinese Checkers}
\author{Group 1: Emil Falk, Rodolphe Lepigre, \\ Salvatore Tomaselli, G\"oran Weinholt}
\maketitle
%-----------------------------------------------------------
\begin{abstract}
Chinese Checkers is a board game for up to six players. The fact that
it's a multi-player game makes it quite different from more widely
explored two-player games like Chess. Writing a computer player for
Chinese Checkers will allow us to explore search algorithms and
heuristics. We hope to discover what makes a multi-player game
different from a two-player game and what effect this has on the
quality of a computer player.
\end{abstract}

\section{Introduction}
\subsection{What is the problem you are trying to solve?}

We want to write a computer program that plays a game. Chinese
Checkers is a board game that can be played by two to six players.
Each player controls ten pegs that are placed in the corners of a
hexagonal star. The board has little holes in it where the pegs can be
placed. In each round of the game a player must move a peg in one of
two ways. A peg can be moved to one of the six holes next to it or, if
one of the holes is occupied by another peg, it can also ``jump'' over
that peg in a straight line (thereby traveling two holes over). The
player can jump over other pegs an unlimited number of times every
round, potentially moving a peg straight across the whole board. The
objective is to move all pegs to the opposing corner.

\subsection{What results are available in the literature on that problem?}
%(describe briefly the scientific papers you found relevant to the problem, references to sec 4.)

Sturtevant has explored the max$^n$ and paranoid algorithms in
relation to Chinese Checkers
\cite{springerlink:10.1007/978-3-540-40031-8_8}. The max$^n$ algorithm
is a generalisation of minimax to $n$-player games. The paranoid
algorithm uses the idea that all other players have formed a coalition
(which means there are only really two players). Later on he also
evaluated the UCT algorithm
\cite{springerlink:10.1007/978-3-540-87608-3_4}, which is described as
``Monte-Carlo-like'' and quite effective against max$^n$, but
requiring a lot of computation time.

An interesting point that Sturtevant makes is that multi-player games
(i.e.~Chinese Checkers) are difficult for computers for two reasons:
search strategies are less effective than for two-player games and
there is a need for opponent modelling which is normally not required.

Huang describes a contest in which he supervised students who wrote
programs to play Chinese Checkers \cite{Huang:2001:SGP:378593.378708}.
The students used iter\-ative-deep\-ening search and experimented with
heuristics. We believe that this will be a good way of starting our
project.

Ulfhake reports on a program for two-player Chinese Checkers
\cite{ulfhake}. The program has a variety of search algorithms
available, including different variants of alphabeta and minimax
search. Something particularly interesting to us is her description of
the heuristics she used. The program was reported to play excellently
against human players. Ulfhake also mentions that finding literature
on computer programs playing Chinese Checkers was very difficult.

Hashavit and Markovitch evaluated the Max-Prob algorithm
\cite{Hashavit}. The algorithm is similar to max$^n$ and paranoid, but
in Chinese Checkers it is significantly better. It computes at each
step which move will be most likely to lead to a winning outcome. It
computes this for all players and maximizes its own probability of
winning.

%% Research into programs that play games is likely to be geared towards
%% producing strong programs that one day will be able to beat the best
%% human players. Perhaps a weaker program would be just as suitable for
%% our needs.

\subsection{What tools and programs are already available for the problem, or for closely related ones?}
%(describe these briefly. Say how you can use them, and how your work will build on them, or differ from them).

\emph{cheech} is a free software implementation of Chinese Checkers.
It provides a graphical interface and a network protocol. It initially
appeared to us that we could make our own program connect to
\emph{cheech} as a player and thereby use the existing multi-player
functionality and graphics. Unfortunately the program has suffered
significant bit rot. We evaluated half a dozen alternatives and found
that none of them still work properly. That being said, we can copy
the architectural ideas used in \emph{cheech}.

\section{Key ideas and delimitations}
\subsection{Why is the problem interesting or significant?  What will a solution achieve?}

An interesting aspect of Chinese Checkers is that it's a multi-player
game. This means that players can form temporary coalitions against
other players. While traditional two-player strategies can be used to
play the game they are not going to be able to play it perfectly and
in this sense Chinese Checkers is still an open problem. Working on an
open problem is always interesting.

The multi-player attribute will provide a good setting for objectively
evaluating the performance of different algorithms and heuristics. The
existing literature is somewhat sparse on how existing computer
programs fare against humans in multi-player Chinese Checkers, and
maybe we can give a more detailed account.

\subsection{The central idea in your solution}

The architecture of our solution will consist of three parts: a game
server written in Erlang, a graphical game client written in Python
and several autonomous game clients. A network protocol has been
specified and the server and the graphical client are already well
under way.

Initially our program will be playing a two-player variant of Chinese
Checkers. When it is working properly we will try with more players.
We will be using an iterative development model that emphasizes
getting the basics up and running quickly.

Our first computer player will be an implementation of \emph{iterative
  deepening depth-first-search} (IDDFS) \cite{aimodern}. This will be
combined with heuristics to form a kind of informed search. To improve
the depth of our solution we will try and do some pruning of ``bad''
plys, e.g.~those that do not move the player any closer to the goal
state. The heuristics described in \cite{ulfhake} will be very useful
for this task. Our first heuristic can be simple: some board
evaluation function based on how close pegs are to the goal, and a
disincentive to leave pegs in the starting position.

The literature describes algorithms that are more capable than
\emph{IDDFS}. An iterative development model means that we will be
making continual improvements to our program, and it is likely that we
will attempt to use some of the more capable algorithms later on. Our
first programs can be preserved for later programs to play against.

\subsubsection{Why do you think it will work?}

Our solution will work because we think that \emph{IDDFS} closely
relates to how a human might play the game: first evaluating moves
that are easily visible on the board and iteratively looking deeper
(further) into what might happen.

Even though \cite{Huang:2001:SGP:378593.378708} does not report on the
performance of \emph{IDDFS} we know that it at least has been used to
some degree of success. The performance of our solution will of course
depend heavily on finding good heuristics. The first program might not
win any contests but it will be able to play the game, because our
heuristics can force it to make forward progress.

\subsection{Define an instance of the problem. How will you measure the performance of your program?}
%(You must know this in advance, find instances /benchmarks, or make your own).

The program should play a game in the same way a human would play: the
program waits its turn and makes a move. The initial board setup is
one instance of the problem, but from the perspective of analysing the
game every intermediate state of the board may also be considered an
instance.

We want to evaluate the program's performance by having it play
against ourselves and other bots.


\subsection{The scope of your work}
%(also, what interesting and related things are outside the scope of your proposal?)

The scope of our project will ultimately be to try and challenge a set
of amateur human players. The first version of the program will play a
two-player game and this will later be extended to a multi-player
game. In developing the program we will explore how the dynamics of
the game changes in the multi-player version. Our program will be able
to play against itself, against humans, and against other programs. An
evaluation of the different versions of the program will be written.

Playing a perfect game must be outside our scope because it appears to
be an open research problem.

\section{Notes on strategies}

When we have the basic mechanics of the game and the computer player
working there are some strategies that we should consider. Some of the
strategies can be implemented by adjusting the heuristic function and
some can be implemented \emph{ah hoc}. The following are some ideas
for strategies.

\textbf{Leave no man behind}. If this strategy is not used it is
possible for the computer player to end up with a lot of pegs very far
away from the goal. These pegs will be very costly to move toward the
goal if it's late in the game, so a heuristic should take that into
account.

\textbf{Stay the path}. The center of the board is more likely to
contain many paths towards the goal. Therefore if the heuristic gives
a preference to moves that keep pegs in the middle then the game is
likely to move faster and give more opportunities to jump long
distances.

\textbf{Forge ahead} or \textbf{the best offense is a good defense}.
The computer player can either be aggressive and try to create long
paths for moving fast and far, or it can be defensive and try to not
give any other players opportunities to use its pegs as part of a
path.

\newpage
\section{First results}
%(A small piece of pseudo-code for a key part of the solution. )

\subsection{Iterative deepening depth-first-search}
A central part of our solution will be the \emph{IDDFS} algorithm.
Shown below is the
high-level pseudocode for that algorithm \cite{aimodern}.\\
\begin{algorithmic}
\Function{ITERATIVE-DEEPENING-SEARCH}{\emph{board, time-limit}} \\
\textbf{global} \emph{best-move} \\
\emph{best-move}.SCORE = 0 \\
\emph{best-move}.MOVE = UNDEFINED \\
\For{\emph{depth} $= 0 \to \infty$}
\State RECURSIVE-DLS(MAKE-NODE(\emph{board}), \emph{depth})
\If {TIME-OUT(\emph{time-limit})}
\State \Return \emph{best-move}
\EndIf
\EndFor
\EndFunction
\\
\Function{RECURSIVE-DLS}{\emph{node, limit}}
\If{\emph{limit} $>$ 0}
\ForAll{\emph{move} in VALID-MOVES(\emph{node})}
\If{\emph{move}.SCORE $>$ \emph{best-move}.SCORE}
\State \emph{best-move} $=$ \emph{move}
\EndIf
\State \emph{child} $=$ CHILD-NODE(\emph{node, move})
\State RECURSIVE-DLS(\emph{child, limit - 1})
\EndFor
\EndIf
\EndFunction
\end{algorithmic}

The algorithm works by iteratively deepening the search tree to find the best possible move in the given
amount of time it has. It does so by recursive calls to RECURSIVE-DLS with an increased depth-limit to expand
the search tree. When the given time-limit is exceeded the algorithm will stop and return the best move it has
found so far. The score of a move is computed using different heuristics.

\bibliographystyle{aiaa}
\bibliography{refs}
% \section{References}

% Reminder: References should be cited (i.e. actually be referred to) at the
% appropriate place in your text, they should visibly influence your document,
% and they should convey as much information as possible to the reader.

% To help you produce a good literature survey, we advise you to use an annot-
% ated list of references.

% A reference to a paper will start with a unique ID (numbers, letters, ...)
% and have the following three paragraphs:
% \begin{enumerate}
% \item Author(s) [year of publication], ``Title of paper'', Journal (or conference etc.), location data (i.e., volume numbers, etc. needed to find the item), a live www link if available.
% \item A couple of lines saying what question the paper is attacking, and what the proposed solution is. Or what topic the reference deals with (your textbook has excellent examples in the sections entitled ``Bibilographical and Historical notes'').
% \item A pointer back to where you cite this reference.
% \end{enumerate}

% An example would look like this:
% \newline
% [REF1]
% \begin{enumerate}
% \item Turing, A.M. (1936).`On Computable Numbers, with an Application to the Entscheidungsproblem". Proceedings of the London Mathematical Society. 2 42: 23065.
% \item The paper is about ...
% \item The most important reference to this paper in our document is in Section XYZ
% \end{enumerate}

% A reference to a specific topic in a book (or large survey article) has similar 2nd. and 3rd. paragraphs, but the first will look like this:
% \begin{enumerate}
% \item Author(s) [year of publication], ``Title of book'', Publisher, a live www link if available.
% \textbf{Section numbers or page numbers in the book dealing with the material you are referring to}.
% \end{enumerate}

% \section{Appendix}
% The first 3 sections (excluding title page and references) are limited to 8 pages. Here, you can put in anything that didn't fit in there, but that you still think is essential.
\end{document}
