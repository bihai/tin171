\section{Results and Evaluation}
% (What does your running code do? How does it fare against your
% benchmarks and instances? Describe advantages and disadvantages,
% possibly in relation to other groups in this course.)

%% To be able to test different approaches, the bot supports multiple
%% functions. In general, Python is a slow interpreted language, but
%% there are projects that aim at providing just in time compilation. The
%% PyPy interpreter has proven to be somewhat faster than the standard
%% CPython. When PyPy is forced to compile loops, after a few iterations,
%% the IDDFS algorithm is able to compute one level deeper. Because the
%% algorithm is time-limited this gives the bot running on PyPy an
%% advantage over the bots running on CPython.

%% We have a time-limited implementation of iterative-deepening depth
%% first search based on an algorithm from \cite{aimodern}. When the
%% algorithm compares two moves that achieve an equally good outcome it
%% will explicitly favor the move that achieves the outcome sooner (at a
%% lower search depth). If this check is omitted the algorithm can end up
%% looping at the end of a game. This is because it would make a move
%% that would be the winning move two moves into the future, but on the
%% next turn it might again choose a move that wins two moves into the
%% future, \emph{ad infinitum}.

We have quite a number of different bot personalities and it is not
possible to test them all against each other in every possible
setting. We have therefore limited ourselves to having each bot play
every other bot in a two-player game. The results are shown in figures
\ref{stats1} and \ref{stats2}. It takes more than 24 hours to run
these simulations, so we have unfortunately not been able to do as
many runs as we would have liked. We had two primary runs of every
combination. For some combinations we also had earlier results that we
were able to integrate with these.

\begin{figure}
\centering
\begin{tabular}{ | c |  c  c  c  c  c  c  c  | }
\hline
   &  0   &  1   &  2   &  3   &  4   &  5   &  6   \\
\hline
0  &      & 1.00 &  .50 &    - &    - &  .00 &  .00 \\
1  &  .00 &      &  .00 &  .50 &    - &  .50 &  .00 \\
2  &  .50 & 1.00 &      & 1.00 &    - &  .50 &  .00 \\
3  & 1.00 & 1.00 & 1.00 &      &    - &  .50 &  .00 \\
4  &    - & 1.00 &    - &    - &      &    - &    - \\
5  & 1.00 &  .75 & 1.00 & 1.00 &    - &      &  .00 \\
6  & 1.00 & 1.00 & 1.00 & 1.00 &    - & 1.00 &      \\
7  & 1.00 & 1.00 &  .00 & 1.00 &    - & 1.00 &  .00 \\
8  & 1.00 &  .50 &  .50 & 1.00 &    - & 1.00 &  .00 \\
9  & 1.00 &  .50 &  .50 & 1.00 &    - & 1.00 &  .00 \\
10 & 1.00 & 1.00 &    - & 1.00 &    - & 1.00 &  .50 \\
11 & 1.00 &  .75 & 1.00 & 1.00 &    - &  .50 & 1.00 \\
12 & 1.00 & 1.00 &    - & 1.00 &    - & 1.00 &  .50 \\
13 & 1.00 & 1.00 & 1.00 & 1.00 &    - & 1.00 & 1.00 \\
14 & 1.00 & 1.00 &    - & 1.00 &    - & 1.00 &  .00 \\
15 &  .50 &  .50 &  .00 &  .50 &    - &  .00 &  .00 \\
\hline
\end{tabular}
\caption{This table shows the proportion of wins in matches between
  two bots. The bots at the left made the first move. Games that did
  not finish are denoted with -. Continued in figure \ref{stats2}.}
\label{stats1}
\end{figure}

\begin{figure}
\centering
\begin{tabular}{ | c |  c  c  c  c  c  c  c  c  c  | }
\hline
   &  7   &  8   &  9   &  10  &  11  &  12  &  13  &  14  &  15  \\
\hline
0  & 1.00 &  .00 &  .00 &  .00 &  .00 &  .00 &  .00 &  .00 & 1.00 \\
1  &  .50 &  .00 &  .50 &  .50 & 1.00 &  .00 &  .50 &  .00 &  .50 \\
2  & 1.00 & 1.00 &  .00 &  .00 &  .00 &  .00 &  .00 &  .00 &  .50 \\
3  & 1.00 &  .50 & 1.00 &  .00 &  .00 &  .50 & 1.00 &  .00 &  .50 \\
4  &    - &    - &  .50 &  .50 &    - &  .00 &    - & 1.00 & 1.00 \\
5  & 1.00 &  .50 &  .50 &  .50 &  .00 &  .00 & 1.00 &  .50 & 1.00 \\
6  & 1.00 &  .50 & 1.00 & 1.00 & 1.00 &  .50 & 1.00 &  .50 & 1.00 \\
7  &      & 1.00 &  .50 &  .00 &  .00 &  .00 & 1.00 &  .00 & 1.00 \\
8  & 1.00 &      &  .00 &  .00 &  .00 &  .50 & 1.00 &  .00 &  .50 \\
9  & 1.00 &  .50 &      & 1.00 &  .50 &  .50 & 1.00 &  .50 & 1.00 \\
10 &    - & 1.00 & 1.00 &      & 1.00 &  .50 &    - & 1.00 & 1.00 \\
11 &  .00 & 1.00 & 1.00 &  .50 &      &  .00 & 1.00 &  .00 & 1.00 \\
12 & 1.00 & 1.00 & 1.00 &  .50 & 1.00 &      & 1.00 &  .50 & 1.00 \\
13 & 1.00 & 1.00 &  .50 &  .00 & 1.00 &  .00 &      &  .00 & 1.00 \\
14 & 1.00 & 1.00 &  .50 & 1.00 & 1.00 &  .50 & 1.00 &      &  .50 \\
15 & 1.00 &  .50 &  .50 &  .00 &  .50 &  .00 &  .50 &  .00 &      \\
\hline
\end{tabular}
\caption{The bot numbers (0-15) are combinations of search algorithms
  and board evaluation functions as shown in figure \ref{botid}.}
\label{stats2}
\end{figure}


\begin{figure}
\centering
\begin{tabular}{|c | c c |}
\hline
  & Search algorithm & Board evaluation function \\
 \hline
0 & Trivial & Euclidean \\
1 & Trivial & Handtuned \\
2 & IDDFS & Euclidean \\
3 & IDDFS & Handtuned \\
4 & MiniMax & Handtuned \\
5 & Parallel & Handtuned \\
6 & Trivial & Evolved \\
7 & IDDFS & Evolved \\
8 & Parallel & Euclidean \\
9 & Parallel & Evolved \\
10 & AlphaBeta & Handtuned \\
11 & AlphaBeta & Evolved \\
12 & Trivial & Evolved3 \\
13 & IDDFS & Evolved3 \\
14 & AlphaBeta & Evolved3 \\
15 & Parallel & Evolved3 \\
\hline
\end{tabular}
\caption{Bot algorithms used in figure \ref{stats1} and \ref{stats2}.}
\label{botid}
\end{figure}



%%\subsection{Bot versus human player}

The goal of our project was to propose machine players that could beat
a novice human player. Among our bots, only a few can pretend to beat
an human player, and we believe that none is really equipped to play
against an expert player. The bots that have proven to be the most
effective against ourselves are:

\begin{itemize}
  \item the \textbf{Alpha-Beta bot}, even if it sometimes leaves a peg behind.
        This bot uses our hand-tuned board evaluation function.
  \item the \textbf{Evolved Alpha-Beta bot}, which uses the board evaluation
        function generated with the genetic algorithm.
\end{itemize}

When playing against these bots you easily see that it tries to adapt and avoid
moves that will open up paths for the opponent and it seldom moves pieces alone.
We tried to take some notes about their
behaviour and how difficult it was to play against them.

Several of the bots were ``cheating'' (i.e.~were not getting out of their
nest fast enough, thus preventing our pegs to get in), in particular the
\textit{IDDFS bots}.

Some bots lost because of pegs left behind: the \textit{minimax bot}, the
\textit{parallel hand-tuned bot} and the \textit{alpha-beta bot}.

The bots that were the easier to win against were the \textit{trivial Euclidean bot} and
the \textit{trivial evolved bot}. The \textit{trivial hand-tuned bot} was easy
to beat as well, but it was a bit more challenging than we expected.

Several of the bots also make very bad decisions, in particular the
\textit{parallel bots}. Sometimes they are just moving inside their
nest even though they have pegs far behind.

The bot that performs the best against one human player is the
\textit{evolved alpha-beta bot}.
