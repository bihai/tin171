\section{Results and Evaluation}
% (What does your running code do? How does it fare against your
% benchmarks and instances? Describe advantages and disadvantages,
% possibly in relation to other groups in this course.)

%% To be able to test different approaches, the bot supports multiple
%% functions. In general, Python is a slow interpreted language, but
%% there are projects that aim at providing just in time compilation. The
%% PyPy interpreter has proven to be somewhat faster than the standard
%% CPython. When PyPy is forced to compile loops, after a few iterations,
%% the IDDFS algorithm is able to compute one level deeper. Because the
%% algorithm is time-limited this gives the bot running on PyPy an
%% advantage over the bots running on CPython.

%% We have a time-limited implementation of iterative-deepening depth
%% first search based on an algorithm from \cite{aimodern}. When the
%% algorithm compares two moves that achieve an equally good outcome it
%% will explicitly favor the move that achieves the outcome sooner (at a
%% lower search depth). If this check is omitted the algorithm can end up
%% looping at the end of a game. This is because it would make a move
%% that would be the winning move two moves into the future, but on the
%% next turn it might again choose a move that wins two moves into the
%% future, \emph{ad infinitum}.

We have quite a number of different bot personalities and it is not
possible to test them all against each other in every possible
setting. We have therefore limited ourselves to having the bots play a
small set of games against a mixed set of opponents. If a bot is any
good it should be able to win against an opponent, no matter exactly
what type of opponent it is. The table in figure \ref{stats} should be
taken with a grain of salt. It shows the percentage of wins the bots
had.

\begin{figure}
\centering
\begin{tabular}{ | l | c | c | c | c | c | }
\hline
Bot & \multicolumn{5}{|c|}{Players} \\
\hline
& 2 & 3 & 4 & 5 & 6 \\
\hline
Trivial-Euclidean & 42\% & 20\% & 50\% & - & 25\% \\
\hline
Trivial-Handtuned & 15\% & - & 60\% & - & 8\% \\
\hline
Trivial-Evolved & 83\% & - & 12\% & - & - \\
\hline
Trivial-Evolved3 & 100\% & - & 12\% & - & - \\
\hline
IDDFS-Euclidean & 50\% & 100\% & - & - & - \\
\hline
IDDFS-Handtuned & 66\% & - & - & - & - \\
\hline
IDDFS-Evolved & - & - & 50\% & - & - \\
\hline
MiniMax-Handtuned & 66\% & - & - & - & - \\
\hline
AlphaBeta-Handtuned & 50\% & - & - & - & - \\
\hline
AlphaBeta-Evolved & 100\% & - & 33\% & - & - \\
\hline
AlphaBeta-Evolved3 & 50\% & - & 50\% & - & - \\
\hline
Parallel-Handtuned & 50\% & 66\% & 25\% & - & - \\
\hline
Parallel-Evolved & 50\% & - & - & - & - \\
\hline
Parallel-Evolved3 & 25\% & - & - & - & - \\
\hline
\end{tabular}
\caption{Portion of wins in a selected set of games.}
\label{stats}
\end{figure}


%%\subsection{Bot versus human player}

The goal of our project was to propose machine players that could beat
a novice human player. Among our bots, only a few can pretend to beat
an human player, and we believe that none is really equipped to play
against an expert player. The bots that have proven to be the most
effective against ourselves are:

\begin{itemize}
  \item the \textbf{Alpha-Beta bot}, even if it sometimes leaves a peg behind.
        This bot uses our hand-tuned board evaluation function.
  \item the \textbf{Evolved Alpha-Beta bot}, which uses the board evaluation
        function generated with the genetic algorithm.
\end{itemize}

When playing against these bots you easily see that it tries to adapt and avoid
moves that will open up paths for the opponent and it seldom moves pieces alone.
We tried to take some notes about their
behaviour and how difficult it was to play against them.

Several of the bots were ``cheating'' (i.e.~were not getting out of their
nest fast enough, thus preventing our pegs to get in), in particular the
\textit{IDDFS bots}.

Some bots lost because of pegs left behind: the \textit{minimax bot}, the
\textit{parallel hand-tuned bot} and the \textit{alpha-beta bot}.

The bots that were the easier to win against were the \textit{trivial Euclidean bot} and
the \textit{trivial evolved bot}. The \textit{trivial hand-tuned bot} was easy
to beat as well, but it was a bit more challenging than we expected.

Several of the bots also make very bad decisions, in particular the
\textit{parallel bots}. Sometimes they are just moving inside their
nest even though they have pegs far behind.

The bot that performs the best against one human player is the
\textit{evolved alpha-beta bot}.
