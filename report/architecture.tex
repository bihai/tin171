\section{Overview of the architecture}
% (Describe the different parts of your program suite in detail.)
%% arena  cclient   gui_client
%% bot    chinese   hs-cli

Our project is divided into a collection of programs that each solve
different parts of the problem. The programs communicate using a
network protocol that we have designed. Our choice to use a network
protocol has let us write the programs in different programming
languages. We have one server implementation, a graphical client, a
command-line client and several bot clients.

Players connect to a \emph{game server} which maintains the current
game state, notifies each player of their turn and checks that the
rules of the game are not violated. A player can host a game or join a
game someone else is hosting. It is also possible to join as a
spectator and see what is going on in a game. If the game is over the
server also announces who the winner was. The server is written in
Erlang and can host multiple games simultaneously.

The \emph{graphical client} is written in Python and is designed for a
human player. The main part of the interface is the current board
state, shown as the star that one would normally see a on real life
version of the game. When it is the players' turn they can make their
move by clicking on the pegs shown on the screen.

The \emph{command-line client} has been written in C. It can be used to
host, join or spectate a game. The board is represented using ASCII
characters, and the holes in the board are indexed in a grid. The aim
of this client was to serve as a base for an efficient implementation of
our bots. However, our Scheme and Haskell implementation have proven to
be fast enough, so the idea was left out.

The \emph{bots} is the programs that contains the AI components of our
project. They are written in Python, Scheme and Haskell respectively.

The \emph{Python bot} is divided in three major components: the protocol
module, the board module and the bot module. The \emph{protocol module}
serializes and deserializes the network protocol. The \emph{board module}
contains code to generate all moves a player has available and it also
contains our board evaluation functions. The \emph{bot module} contains
the search algorithms. It is the job of the search algorithms to use the
possible moves and the board evaluation functions to determine which move
will give the best outcome for the bot. When we have implemented a new
search algorithm or board evaluation function the previous ones are still
available. Each combination of search algorithm and evaluation function
is given a \emph{bot personality} identifier.

The \emph{Haskell bot} ... % TODO

The \emph{Scheme bot} ... % TODO

An important part of our project is not only to implement search
algorithms and heuristics, but to also measure their performance. To
aid in this we have the \emph{arena} program. This program can start
up a game between different bot personalities so that we can determine
which bot is the strongest in different situations.

