\section{Discussion and Conclusions}
%(Sum up your project, suggest future extensions and improvements.)

Our computer program can win a game against a human. Our goal has been
fulfilled and many ideas for future development have been brought to
light. The multi-player side of Chinese Checkers remains largely
unexplored for us but it is still very interesting while in the
meantime being rather complicated.

One of our most interesting discoveries was about the genetic
algorithm. It was surprisingly effective in finding heuristic
functions. It would be interesting to see how it would be affected by
variants of the fitness function, e.g.~making the generations play
against several different bots. Games with five and six players have
not yet been explored.

It turned out that two-player games are easier to handle than
multiplayer games for a number of reasons. First of all in two-player
games there are fewer pegs on the board and therefore fewer possible
moves. This means the search space is that much smaller. Add a few
players and not only are there more possible moves, but now the
minimax algorithm is no longer sufficient (max$^n$ or something more
advanced must be used). Alpha-beta pruning is no longer effective
either \cite{Korf199199}.

Algorithms for solving multi-player games have been discussed recently
in \cite{Hashavit} and \cite{bestreplysearch}, but we have not found
any discussion about them in the literature. However investigating
them would be a very natural continuation of this project.
