\section{Discussion and Conclusions}
\subsection{Bot VS human player}
The goal of our project was to propose machine players that could beat a
novice human player. Among our bots, only a few can pretend to beat an human
player, and none is really equiped to play against en expert player. The bots
that have proven to be the most effective against an human player are:

\begin{itemize}
  \item the \textbf{Alpha-Beta bot}, even if it sometimes leave a peg behind.
        This bot uses our hand-tuned board evaluation function.
  \item the \textbf{Evolved Alpha-Beta bot}, which uses the board evaluation
        function generated with the genetic algorithm.
\end{itemize}

When playing against those bots you easily see that it tries to adapt, avoid
moves that will open up paths for the opponent and seldom move pieces alone.

While playing against our bots we tried to take some notes about their
behaviour, and how difficult it was to play against them.

Several of the bots were ``cheating'' (i.e. were not getting out of their
nest fast enough, thus preventing our pegs to get in), in particular the
\textit{IDDFS bots}.

Some bots lost because of pegs left behind: the \textit{minimax bot}, the
\textit{parallel static bot} and the \textit{alpha-beta bot}.

The bots that were the easyer to win against were the \textit{trivial bot} and
the \textit{evolved distance bot}. The \textit{static distance bot} was easy
to beat also, but it was a bit more chalenging (more than we expected).

Several of the bots also take very bad decisions, in particular the \textit{
parallel bots}. Sometimes they are just moving inside their nest though they
have pegs far behind.

The bots that performs the best against one human player is the \textit{
evolved alpha-beta bot}.

% TODO add things about multi-player

\subsection{Bot VS Bot}
A result that we have not figured out is why our bots with lookahead is
getting beaten by the trival ones. A hypothesis is that \emph{minimax} is modelling
its opponent as always doing the right thing, but since the trival bots only
looks at what is right for the moment it does not behave as the minimax
bot thinks it should. And this then leads it to making the wrong decisions.

\subsection{Conclusion}
Our goal have been fulfilled, and many ideas for future development have
arised. The multi-player side of Chinese Checkers remains largely unexplored
for us but it is still very interesting while in the meantime being rather
complicated.

Algorithms for solving multi-player games have been discussed recently in
\cite{Hashavit} and \cite{bestreplysearch}, but there have not been much
discussion about them in the scientific world. However investigating them
would be a very natural continuation for this project.

One of our most interesting discovery was about the genetic algorithm, and
it would be nice to see how far it could get by making the generations play
against several different bots. The game with 5 and 6 players remains to be
explored completely.
% TODO

%(Sum up your project, suggest future extensions and improvements.)
