\section{Discussion and Conclusions}
\subsection{Bot VS human player}
The goal of our project was to propose machine players that could beat a
novice human player. Among our bots, only a few can pretend to beat an human
player, and none is really equiped to play against en expert player. The bots
that have proven to be the most effective against an human player are:
\begin{itemize}
  \item the \textbf{Alpha-Beta bot}, even if it sometimes leave a peg behind.
        This bot uses our hand-tuned board evaluation function.
  \item the \textbf{Evolved Alpha-Beta bot}, which uses the board evaluation
        function generated with the genetic algorithm.
\end{itemize}

When playing against those bots you easily see that it tries to adapt, avoid
moves that will open up paths for the opponent and seldom move pieces alone.

% TODO add a table with stats
% TODO add things about multi-player

\subsection{Bot VS Bot}
One thing we still quite havn't figured out is why our bots with lookahead is
getting beaten by the trival ones. A hypothesis is that minimax is modelling
it's opponent as always doing the right thing but since the trival bots only
looks at what's right for the moment it doesn't behave as the minimax/alphabeta
bot thinks it should. And this then leads it to making the wrong decisions.

\subsection{Conclusion}
Our goal have been fulfilled, and many ideas of futur developpement have
arised. The multi-player side of Chinese Checkers make it very interesting,
but in the meantime very complicated.

Algorithms for solving multi-player games have been discussed recently in
\cite{Hashavit} and \cite{bestreplysearch}, but there have not been much
discussion about them in the scientific world. However investigating them
would be a very natural continuation for this project.

One of our most interesting discovery was about the genetic algorithm, and
it would be nice to see how far it could get by making the generations play
against several different bots. The game with 5 and 6 players remains to be
explored completely.
% TODO

%(Sum up your project, suggest future extensions and improvements.)
