%% \subsection{Parallel bot}
%% \label{parallel}

Our search algorithms presented thus far have been designed to run on
single processors. We have given the algorithms a limited time in
which to pick a move, and therefore a faster search algorithm might
perform better. Modern machines have multiple processor cores, so we
have written a parallel bot that uses those cores.

%% This is a sideline and does not belong in this part of the report:

%% The commonly used Python implementation CPython has a global interpreter lock
%% which prevents multiple threads to really running in parallel when the machine
%% provides more than one thread/core \cite{pygil}. To circumvent this limitation
%% Python provides a module to distribute computation over several processes,
%% providing shared memory and other means of comunication.

The parallel bot can also be combined with different board evaluation functions.
Its first step is always to
evaluate the boards resulting from the possible moves it can do. This is so that in case of a
timeout it will always have a move to perform.
The different outcomes are sorted and new processes are then spawned to
evaluate the future scenarios deriving from the best immediate moves.
The child processes will continue to evaluate future moves until they are
terminated.
An overview of the algorithm is shown in figure \ref{parallel_algorithm}.

\begin{figure}
\begin{algorithmic}
\Function{Parallel-Bot}{board, player-id}
    \State moves = \textsc{All-possible-Moves}
    \State points = \textsc{Sort}(\textsc{Evaluate}(moves))
    \State best = points[0]
    \State promising = points[$0\dots 5$]
    \ForAll{move $\in$ promising}
        \State child = \textsc{Fork}(\textsc{Move}, player-id)
        \State \textsc{Append}(children, child)
    \EndFor
    \State \textsc{Wait}(timeout)

    \ForAll{C $\in$ children}
        \State \textsc{Terminate}(C)
        \If {C.result $>$ best}
            \State best = C.result
        \EndIf
    \EndFor

    \State \Return best
\EndFunction
\\
\Function{Move}{player-id}
    \State result = 0
    \While{$\top$}
        \ForAll{P $\in$ opponents}
            \State boards = \textsc{Evaluate-Possible-Boards}()
            \State boards = \textsc{Prune}(boards)
            \ForAll{B $\in$ boards}
                \If {\textsc{Evaluate}(B) $>$ result}
                    \State result = \textsc{Evaluate}(B)
                \EndIf
            \EndFor
        \EndFor
    \EndWhile
\EndFunction

\end{algorithmic}
\caption{Overview of the parallel algorithm.}
\label{parallel_algorithm}
\end{figure}
