\subsection{Parallel bot}
\label{parallel}

The approach for the parallel bot is to try to use multiple cores on systems to
try to achieve better results in evaluating moves within the time constraints.

The commonly used Python implementation CPython has a global interpreter lock
which prevents multiple threads to really run in parallel when the machine
provides more than one thread/core\cite{pygil}. To circumvent this limitation
Python provides a module to distribute computation over several processes,
providing shared memory and other means to comunicate.

The parallel bot can use different heuristics. Its first step is always to
evaluate the boards resulting from the possible moves it can do. So in case of a
timeout it will always have a move to perform.
The outcomes of the board are sorted and new processes are then spawned to
evaluate the future scenarios deriving from the best immediate moves.
The child processes will continue evaluate moves in the future until they are
terminated.
The algorithm is shown in detail in Figure \ref{parallel_algorithm}.

\begin{figure}
\begin{algorithmic}
\Function{parallel-bot}{board, player-id}
    \State Moves = \textsc{All-possible-Moves}
    \State Points = Sort(Evaluate(\textsc{Moves}))
    \State Best = Points[0]
    \State Promising = Points[0..5]
    \ForAll{move $\in$ \textsc{Promising}}
        \State NewChild = Fork(move, player-id)
        \State Child.append(NewChild)
    \EndFor
    \State Wait(timeout)

    \ForAll{c $\in$ Child}
        \State Terminate(c)
        \If {c.result $>$ Best}
            \State Best = c.result
        \EndIf
    \EndFor

    \State \Return Best
\EndFunction
\\
\Function{move}{move, player-id}
    \State Result = 0
    \While{True}
        \ForAll{P $\in$ Opponents}
            \State Boards=EvaluatePossibleBoards()
            \State Boards=Prune(Boards)
            \ForAll{B $\in$ Boards}
                \If {Evaluate(B) > Result}
                    \State Result = Evaluate(B)
                \EndIf
            \EndFor
        \EndFor
    \EndWhile
\EndFunction

\end{algorithmic}
\caption{parallel algorithm}
\label{parallel_algorithm}
\end{figure}