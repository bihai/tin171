
\section{Future work}

Towards the end of the game all the players' pegs are located around
their target corner and there are no longer interactions between them
and other players. In this situation it is not necessary to use a
\emph{minimax} algorithm, because calculating the moves of the
opponents is no longer necessary. The objective becomes to place
all the pegs in the target using the least possible amount of moves.

For this purpose the bots could switch to the \emph{iterative
  deepening depth-first-search} or could even use a database of
pre-computed moves. In the endgame phase a bot could consider a board
where all the pegs of the opponents are removed, and those boards
could be hashed and used as keys for an hash table that matches the
best move for each board configuration.

A similar situation occurs at the start of a game.
Until the players' pegs are close enough to interact with each other, an opening
book of pre-computed moves could be used. This would be useful just for two or
three players games since in all the other configurations players are located in
adjacent corners and can interact with each other already from the
second or third moves.

We investigated the Max-Prob algorithm, designed by Hashavit and
Mar\-ko\-vi\-tch in \cite{Hashavit}. However, some details of the
algorithm being unclear in the paper, we decided not to implement this
algorithm. According to the results presented in \cite{Hashavit}, the
Max-Prob algorithm should lead to better results than any other
algorithms considered in this report for games with three players or
more.

%% The algorithm is based on a quite natural remark: if there are several
%% possible moves leading to the same probability of winning the game, the bot
%% should be as likely to choose any of these moves.

%% To penalize bots that leave pegs behind, it is possible to use the function
%% $d_{1}(x,y) = d(x,y)^{2}$ where \emph{d} is one of the already specified
%% distance functions. In this way the bots will reach better positions by making
%% the last peg to jump forward as far as possible.

A result that we have not figured out is why our bots with lookahead
is getting beaten by the trival ones. A hypothesis is that minimax is
modelling its opponent as always doing the right thing, but since the
trival bots only look at what is right for the moment it does not
behave as the minimax bot thinks it should. This then leads it to
making the wrong decisions.

A better way of benchmarking the bots against each other could be to
use tournaments. This is how the same problem is solved in sports,
where there exists a large number of teams. It is not possible for
every team to play against every other team, just like our bots can't
play against every other bot. In our case it is worse, because we
would really like to see the bots play against more than one player as
well.
