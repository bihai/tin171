
\section{Future work}

\subsection{Abandoned ideas}

\subsubsection{Endgame}
We have observed that in the phase of the endgame all the players' pegs are
located around their target corner, and there are no longer interactions between
them.
In this situation it is no longer necessary to use a \emph{minimax} algorithm,
because calculating the moves of the opponents is no longer necessary, so the
objective becomes to place all the marbles in the target using the least
possible amount of moves.
For the purpose the bots could switch to the \emph{iterative deepening
depth-first-search} or could even use a database of pre-computed moves.
In endgame phase a bot could consider a board where all the pegs of the
opponents are removed, and those boards could be hashed and used as keys for an
hash table that matches the best move for each board configuration.

\subsubsection{Startgame}
Until the players' pegs are close enough to interact with each other, an opening
book of pre-computed moves could be used. This would be useful just for two or
three players games since in all the other configurations players are located in
adjacent corners and can interact with each other already from the
 2\textsuperscript{nd} or 3\textsuperscript{rd} move.

\subsection{The \textit{Max-Prob} algorithm}
We investigated the \textit{Max-Prob} algorithm, designed by Hashavit and
Mar\-ko\-vi\-tch in \cite{Hashavit}. However, some details of the algorithm being
unclear in the paper, we decided not to implement this algorithm.

According to the results presented in \cite{Hashavit}, the \textit{Max-Prob}
algorithm should lead to better results that any other algorithms considered
in this report for games with three players or more.

The algorithm is based on a quite natural remark: if there are several
possible moves leading to the same probability of winning the game, the bot
should be as likely to choose any of these moves.

\begin{figure}
\begin{algorithmic}
\Function{Max-Prob}{board, player-id}
\State Moves = \textsc{All-possible-Moves}
\State P = Max\{WinProb(m, player-id) $| \; c \in$ Moves\}
\State $C_{max}$ = \{$m \in$ Moves $|$ WinProb(m, player-id) = P\}
\State \Return rand($C_{max}$)
\EndFunction
\end{algorithmic}
\caption{Max-Prob algorithm.}
\end{figure}
% TODO Continue


%% To penalize bots that leave pegs behind, it is possible to use the function
%% $d_{1}(x,y) = d(x,y)^{2}$ where \emph{d} is one of the already specified
%% distance functions. In this way the bots will reach better positions by making
%% the last peg to jump forward as far as possible.

\subsection{Bot versus Bot}

A result that we have not figured out is why our bots with lookahead is
getting beaten by the trival ones. A hypothesis is that \emph{minimax} is modelling
its opponent as always doing the right thing, but since the trival bots only
looks at what is right for the moment it does not behave as the minimax
bot thinks it should. And this then leads it to making the wrong decisions.

\subsection{Tournaments}
