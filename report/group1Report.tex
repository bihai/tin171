\documentclass[a4paper,11pt]{article}
\usepackage{fancyhdr}
\setlength{\headheight}{11pt}
\pagestyle{fancyplain}
\usepackage{algpseudocode}
%\renewcommand{\chaptermark}[1]{\markboth{#1}{}}

\lhead{ }
\rhead{}
%\renewcommand{\headrulewidth}{0.0pt}

\lfoot{Group 1: final report}
\cfoot{\thepage}
\rfoot{}

%
%%    homebrew commands -- to save typing
\newcommand\etc{\textsl{etc}}
\newcommand\eg{\textsl{eg.}\ }
\newcommand\etal{\textsl{et al.}}
\newcommand\Quote[1]{\lq\textsl{#1}\rq}
\newcommand\fr[2]{{\textstyle\frac{#1}{#2}}}
\newcommand\miktex{\textsl{MikTeX}}
\newcommand\comp{\textsl{The Companion}}
\newcommand\nss{\textsl{Not so Short}}


\begin{document}
%-----------------------------------------------------------
\title{Final report\\Fun With Chinese Checkers}
\author{Group 1: Emil Falk, Rodolphe Lepigre,\\
        Salvatore Tomaselli, G\"oran Weinholt}
\maketitle
%-----------------------------------------------------------
\begin{abstract}\centering
%%COMMENT (re-use from project proposal and modify if needed)
Chinese Checkers is a board game for up to six players. The fact that
it is a multi-player game makes it quite different from more widely
explored two-player games like Chess. In writing a computer player for
Chinese Checkers we have explored search algorithms and heuristics.
Our aim is to highlight the ways in which multi-player games differ
from two-player games, and the consequences this has for the
effectiveness of a computer player.
\end{abstract}

\section{Introduction and Background}
% (A short general text about the area of ai that your project belongs to.)
Chinese Checkers is a board game that can be played by two to six players.
Each player controls ten pegs that are placed in the corners of a
hexagonal star. The board has little holes in it where the pegs can be
placed. In each round of the game a player must move a peg in one of
two ways. A peg can be moved to one of the six holes next to it or, if
one of the holes is occupied by another peg, it can also ``jump'' over
that peg in a straight line (thereby traveling two holes over). The
player can jump over other pegs an unlimited number of times every
round, potentially moving a peg straight across the whole board. The
objective is to move all pegs to the opposite corner.
% Maybe it would be nice to add a few lines about the history of the game ?

\subsection{The problem you tried to solve}
% (Re-use from project proposal, add changes, refinements,
% extensions. If there has been major changes since the project
% proposal, describe and motivate.)
We have implemented a computer program capable of playing Chinese
Checkers. Our goal has been to create a program that could beat a
novice human player.

\subsection{Related work}
% (Re-use form project proposal, extend with furhter findings from
% literature study, cite and add references. Expain which are relevant
% for your project and which not and why.)
Sturtevant has explored the max$^n$ and paranoid algorithms in
relation to Chinese Checkers
\cite{springerlink:10.1007/978-3-540-40031-8_8}. The max$^n$ algorithm
is a generalisation of minimax to $n$-player games. The paranoid
algorithm uses the idea that all other players have formed a coalition
(which means there are only really two players). Later on he also
evaluated the UCT algorithm
\cite{springerlink:10.1007/978-3-540-87608-3_4}, which is described as
``Monte-Carlo-like'' and quite effective against max$^n$, but
requiring a lot of computation time.

An interesting point that Sturtevant makes is that multi-player games
(i.e.~Chinese Checkers) are difficult for computers for two reasons:
search strategies are less effective than for two-player games and
there is a need for opponent modelling which is normally not required.

Huang describes a contest in which he supervised students who wrote
programs to play Chinese Checkers \cite{Huang:2001:SGP:378593.378708}.
The students used iter\-ative-deep\-ening search and experimented with
heuristics. We believe that this will be a good way of starting our
project.

Hashavit and Markovitch evaluated the Max-Prob algorithm
\cite{Hashavit}. The algorithm is similar to max$^n$ and paranoid, but
in Chinese Checkers it is significantly better. It computes at each
step which move will be most likely to lead to a winning outcome. It
does this for all players and maximizes its own probability of
winning.

Ulfhake reports on a program for two-player Chinese Checkers
\cite{ulfhake}. The program has a variety of search algorithms
available, including different variants of alphabeta and minimax
search. Something particularly interesting to us is her description of
the heuristics she used. The program was reported to play excellently
against human players.

%% Research into programs that play games is likely to be geared towards
%% producing strong programs that one day will be able to beat the best
%% human players. Perhaps a weaker program would be just as suitable for
%% our needs.

\subsection{Tools and programs available}
% (Re-use and extend your list of tools and programs from the project
% proposal including references.
% Which ones are you actually using?  Cite and annotate them.
% Did you encounter any practical difficulties?)
There is a free software implementation of Chinese Checkers called
\emph{cheech}. It provides a graphical interface and a network
protocol. It initially appeared to us that we could make our own
program connect to \emph{cheech} as a player and thereby use the
existing multi-player functionality and graphics. Unfortunately the
program has suffered significant bit rot and we were unable to compile
it. We evaluated half a dozen alternatives and found that none of them
still work properly. That being said, we did copy the server-client
model of \emph{cheech} for our own work.

\section{The central idea in your solution}
% (Re-use the content and structure of subsections from project
% proposal. Add refinments and extensions, motivate changes compared
% to earlier stages of the project.)
\subsection{Significance}

An interesting aspect of Chinese Checkers is that it is a multi-player
game. This means that players can form temporary coalitions against
other players. While traditional two-player strategies can be used to
play the game they are not going to be able to play it perfectly and
in this sense Chinese Checkers is still an open problem. Working on an
open problem is always interesting.

The multi-player attribute provides a good setting for objectively
evaluating the performance of different algorithms and heuristics. The
existing literature is somewhat sparse on how existing computer
programs fare against humans in multi-player Chinese Checkers. We hope
to rectify that with this report.

\subsection{The central idea in our solution}

The architecture of our solution consists of three parts: a game
server written in Erlang, a graphical game client written in Python
and several autonomous game clients. A network protocol has been
specified that supports multiple running games. The server coordinates
the game and tells each player what is happening to the board and
prompts players to make their move. The server verifies that the moves
are legal and checks if there is a winner. The graphical client can
connect either as a player or a spectator. It shows a graphical
representation of the board and lets the user make moves.

%% Initially our program will be playing a two-player variant of Chinese
%% Checkers. When it is working properly we will try with more players.
%% We will be using an iterative development model that emphasizes
%% getting the basics up and running quickly.

Our first computer player is an implementation of \emph{iterative
  deepening depth-first-search} (IDDFS) \cite{aimodern}. This has been
combined with heuristics to form a kind of informed search.

% We
% currently have two heuristics based on distance functions. The first
% is the sum of the Euclidean distance of all pegs from the goal. The
% second is a hand-tuned function that tries to penalize the bot when it
% is leaving pegs behind and attempts to keep pegs in the middle of the
% board, where it is more likely that good paths towards the goal will
% be formed. The heuristics described in \cite{ulfhake} will be very
% useful for our continued work.

\subsubsection{Heuristics}
Functions to evaluate boards must be used to estimate which moves are better
than others, and also to perform pruning.
One idea is to consider the board as a two-dimensional array, use the Euclidean
distance of all the pegs from their target position. Another board evaluation
function that we used uses an hand-tuned distance function that counts the moves
without jumps necessary to reach the corner position in the target.
Currently both functions are implemented and can be used by the bots.
To penalize bots that leave pegs behind, it is possible to use the function
$d_{1}(x,y) = d(x,y)^{2}$ where \emph{d} is one of the already specified
distance functions. In this way the bots will reach better positions by making
the last peg to jump forward as far as possible.
In different test games between the bots, it was noted that while moving to
non-central positions can constitute a disadvantage in two players games, it can
represent a good strategy on more populated boards where the moves available in
the central area are very limited. As a consequence the heuristics that prove to
be effective on one scenario can lead the bot to defeat in other scenarios.
Having said that, a good heuristic for two players games seems to assign more
value to boards that keep the pegs in the central area of the board, as
described in \cite{ulfhake}.
The approach of keeping all the pegs in a cluster to allow jumps and to give low
values to boards where the pegs of a certain player are scattered all around is
a good solution for two players games, but in other cases the board is crowded
so there are several opportunities to perform long jumps even for pegs isolated
from the other ones of the same color.
The three pegs located in the deepest positions of the corner risk to be trapped
by the opponent if they are left behind, so the hand-tuned distance function has
been modified to assign to these positions a great penality.

\subsubsection{Startgame}
Until the players' pegs are close enough to interact with each other, an opening
book of pre-computed moves could be used. This would be useful just for two
players games since in all the other configurations players are located in
adjacent corners and can interact with each other already from the 2nd or 3rd
move.

\subsubsection{Endgame}
We have observed that in the phase of the endgame all the players' pegs are
located around their target corner, and there are no longer interactions between
them.
In this situation it is no longer necessary to use a \emph{minimax} algorithm,
because calculating the moves of the opponents is no longer necessary, so the
objective becomes to place all the marbles in the target using the least
possible amount of moves.
For the purpose the bots could switch to the \emph{iterative deepening
depth-first-search} or could even use a database of pre-computed moves.
In endgame phase a bot could consider a board where all the pegs of the
opponents are removed, and those boards could be hashed and used as keys for an
hash table that matches the best move for each board configuration.

\subsubsection{The Python client}
To be able to test different approaches, the Python client supports multiple
functions.
In general, Python is a slow interpreted language, but there are projects that
aim in providing just in time compilation. The PyPy interpreter has proven to be
somewhat faster than the standard CPython, when forced to compile loops after
few iterations the IDDFS algorithm is able to compute one level deeper.
%% To improve
%% the depth of our solution we will try and do some pruning of ``bad''
%% plys, e.g.~those that do not move the player any closer to the goal
%% state.

The literature describes algorithms that are more capable than
\emph{IDDFS}. An iterative development model means that we will be
making continual improvements to our program, and it is likely that we
will attempt to use some of the more capable algorithms later on. Our
first programs can be preserved for later programs to play against.

%% \subsubsection{Why do you think it will work?}% Not in the report template

%% Our solution will work because we think that \emph{IDDFS} closely
%% relates to how a human might play the game: first evaluating moves
%% that are easily visible on the board and iteratively looking deeper
%% (further) into what might happen.

%% Even though \cite{Huang:2001:SGP:378593.378708} does not report on the
%% performance of \emph{IDDFS} we know that it at least has been used to
%% some degree of success. The performance of our solution will of course
%% depend heavily on finding good heuristics. The first program might not
%% win any contests but it will be able to play the game, because our
%% heuristics can force it to make forward progress.

%% \subsection{Define an instance of the problem. How will you measure the
%%performance of your program?}% Not in the report template
%% %(You must know this in advance, find instances /benchmarks, or make your
%%own).

%% The program should play a game in the same way a human would play: the
%% program waits its turn and makes a move. The initial board setup is
%% one instance of the problem, but from the perspective of analysing the
%% game every intermediate state of the board may also be considered an
%% instance.

%% We want to evaluate the program's performance by having it play
%% against ourselves and other bots.


%% \subsection{The scope of your work}% Not in the report template
%% %(also, what interesting and related things are outside the scope of your
%%proposal?)

%% The scope of our project will ultimately be to try and challenge a set
%% of amateur human players. The first version of the program will play a
%% two-player game and this will later be extended to a multi-player
%% game. In developing the program we will explore how the dynamics of
%% the game changes in the multi-player version. Our program will be able
%% to play against itself, against humans, and against other programs. An
%% evaluation of the different versions of the program will be written.

%% Playing a perfect game must be outside our scope because it appears to
%% be an open research problem.

%% \subsection{Notes on strategies}% not in the report template

%% When we have the basic mechanics of the game and the computer player
%% working there are some strategies that we should consider. Some of the
%% strategies can be implemented by adjusting the heuristic function and
%% some can be implemented \emph{ah hoc}. The following are some ideas
%% for strategies.

%% \textbf{Leave no man behind}. If this strategy is not used it is
%% possible for the computer player to end up with a lot of pegs very far
%% away from the goal. These pegs will be very costly to move toward the
%% goal if it's late in the game, so a heuristic should take that into
%% account.

%% \textbf{Stay the path}. The center of the board is more likely to
%% contain many paths towards the goal. Therefore if the heuristic gives
%% a preference to moves that keep pegs in the middle then the game is
%% likely to move faster and give more opportunities to jump long
%% distances.

%% \textbf{Forge ahead} or \textbf{the best offense is a good defense}.
%% The computer player can either be aggressive and try to create long
%% paths for moving fast and far, or it can be defensive and try to not
%% give any other players opportunities to use its pegs as part of a
%% path.

\section{Overview of the architecture}
% (Describe the different parts of your program suite in detail.)
\subsection{Running modules}
%% arena  cclient   gui_client
%% bot    chinese   hs-cli

Our project is divided into a collection of programs that each solve
different parts of the problem. The programs communicate using a
network protocol that we have designed. Our choice to use a network
protocol has let us write the programs in different programming
languages. We have one server implementation, a graphical client and a
bot client.

%% TODO: note the cli client and the hs client?

Players connect to a \emph{game server} which maintains the current
game state, notifies each player of their turn and checks that the
rules of the game are not violated. A player can host a game or join a
game someone else is hosting. It is also possible to join as a
spectator and see what is going on in a game. If the game is over the
server also announces who the winner was. The server is written in
Erlang and can host multiple games simultaneously.

The \emph{graphical client} is written in Python and is designed for a
human player. The main part of the interface is the current board
state, shown as the star that one would normally see a on real life
version of the game. When it is the players' turn they can make their
move by clicking on the pegs shown on the screen.

The \emph{bot} is the program that contains the AI components of our
project. It is also written in Python and is divided in three major
components: the protocol module, the board module and the bot module.
The protocol module serializes and deserializes the network protocol.
The board module contains code to generate all moves a player has
available and it also contains our board evaluation functions. The bot
module contains the search algorithms. It is the job of the search
algorithms to use the possible moves and the board evaluation
functions to determine which move will give the best outcome for the
bot. When we have implemented a new search algorithm or board
evaluation function the previous ones are still available. Each
combination of search algorithm and evaluation function is given a
\emph{bot personality} identifier.

An important part of our project is not only to implement search
algorithms and heuristics, but to also measure their performance. To
aid in this we have the \emph{arena} program. This program can start
up a game between different bot personalities so that we can determine
which bot is the strongest in different situations.

\subsection{Modules designed but not implemented}

\subsection{Modules a future continuation may have}

\section{Results and Evaluation}
% (What does your running code do? How does it fare against your
% benchmarks and instances? Describe advantages and disadvantages,
% possibly in relation to other groups in this course.)

We have a time-limited implementation of iterative-deepening depth
first search based on an algorithm from \cite{aimodern}. When the
algorithm compares two moves that achieve an equally good outcome it
will explicitly favor the move that achieves the outcome sooner (at a
lower search depth). If this check is omitted the algorithm can end up
looping at the end of a game. This is because it would make a move
that would be the winning move two moves into the future, but on the
next turn it might again choose a move that wins two moves into the
future, \emph{ad infinitum}.

\begin{algorithmic}
\Function{IDDFS}{board, player-id, time-limit}
\State best.move = \emph{nil}
\State best.score = $-\infty$
\State best.depth = $\infty$
\For{depth $= 0 \to \infty$}
 \ForAll{move $\in$ possible-moves(board, player-id)}
  \State nboard = update-board(board, move)
  \State RECURSIVE-DLS(move, nboard, depth, 0, best)
 \EndFor
 \If{timeout}
  \State \Return best.move
 \EndIf
\EndFor
\EndFunction
\\
\Function{RECURSIVE-DLS}{move1, board, limit, depth, best}
\State score = board-evaluation(board, player-id)
\If{score $>$ best.score $\vee$ (score $=$ best.score $\wedge$ depth $<$
best.depth)}
\State best.move = move1
\State best.score = score
\State best.depth = depth
\EndIf
\If{limit $\le$ 0 $\vee$ timeout}
\State \Return
\EndIf
\ForAll{move $\in$ possible-moves(board, player-id)}
 \State nboard = update-board(board, move)
 \State RECURSIVE-DLS(move1, nboard, limit$-1$, depth$+1$, best)
\EndFor
\EndFunction
\end{algorithmic}


%% Fill in with ideas for what to write about:

%% * The multiplayer aspect
%% ** The search-tree explodes with many players
%% ** So what else changes with three or more players?
%% ** What about a bot that recognizes others of its ilk and conspires?

\section{Discussion and Conclusions}
%(Sum up your project, suggest future extensions and improvements.)


% \section{References}
% (Re-use from previous documents, extend.)
\bibliographystyle{aiaa}
\bibliography{refs}

\section{Appendix}
% (Here you include the diary and all other information and
% documentation that did not fit into the report in the above sections
% but that you consider too important to leave out.)

\end{document}
